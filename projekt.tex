\documentclass[10pt,a4paper]{report}
\usepackage[utf8]{inputenc}
\usepackage[T1]{fontenc}
\usepackage{amsmath}
\usepackage{amsfonts}
\usepackage{amssymb}
\usepackage{makeidx}
\usepackage{graphicx}
\usepackage[top=0.5cm]{geometry}
\begin{document}

Zmienna dyskretna
Niech X będzie zmienną losową typu dyskretnego. Wartością oczekiwaną nazywa się sumę iloczynów wartości tej zmiennej losowej oraz prawdopodobieństw z jakimi są one przyjmowane.

Jeżeli dyskretna zmienna losowa X przyjmuje wartości $x_1, x_2, \dots, x_n$ z prawdopodobieństwami wynoszącymi odpowiednio $p_1, p_2, \dots, p_n$, to wartość oczekiwana $\mathbb EX$ zmiennej losowej X wyraża się wzorem
\begin{equation}
\mathbb EX = \sum_{i=1}^n x_i p_i.
\end{equation}
Wariancja – klasyczna miara zmienności.\\
 Intuicyjnie utożsamiana ze zróżnicowaniem zbiorowości; jest średnią arytmetyczną kwadratów odchyleń (różnic) poszczególnych wartości cechy od wartości oczekiwanej.\\
Wariancja zmiennej losowej  X , oznaczana jako  $\operatorname{Var}[X]$  lub  $D^2 (X)$ , zdefiniowana jest wzorem:
\begin{equation*}
\operatorname{Var}[X]=\mathbb E[(X-\mu)^2],
\end{equation*}
gdzie:\\
$\mathbb E[\dots ]$ jest wartością oczekiwaną zmiennej losowej podanej w nawiasach kwadratowych,\\
$\mu$; jest wartością oczekiwaną zmiennej X;.\\
\underline{Przykład}\\
Obliczymy wariancje dla rzutu symetryczną kostką do gry\\
\begin{eqnarray*}
D^2(X)=\mathbb E[(X-\mathbb EX)^2]=\mathbb E(X^2)- (\mathbb EX)^2=\\
=\sum_{i=1}^{6}\frac{i^2}{6}-(\sum_{i=1}^{6}\frac{i}{6})^2=\frac{1+4+9+16+25+36}{6}\\
-(\frac{1+2+3+4+5+6}{6})^2=\frac{91}{6}-\frac{49}{4}=\frac{35}{12}
\end{eqnarray*}

\end{document}